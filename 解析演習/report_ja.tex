\documentclass{jarticle}
\usepackage[dvipdfmx]{graphicx}
\usepackage{overpic}
\usepackage[margin=2cm]{geometry}
\title{解析演習最終レポート}
\author{土木 太郎}
\date{\today}
\begin{document}
\maketitle

\section{フレーム実験の説明}
フレーム実験の概要を書く.

\section{骨組み構造解析による数値計算と実験の比較}
数値計算と実験で得られた結果を比較する.必要があれば表を用いる.
また,数値計算と実験の値が異なる理由を考察する.
 \begin{table}[h]
  \caption{数値計算と実験の比較}
  \centering
 \begin{tabular}[t]{c|cc}\hline
  観測点&数値解析 ($\mu\varepsilon$) & ~実~~験~($\mu\varepsilon$)\\\hline
A  & 10.0& 11.0\\
B  & 11.0& 12.0\\
C  & 12.0& 13.0\\
D  & 12.0& 13.0\\
E  & 12.0& 13.0\\
F  & 12.0& 13.0\\\hline
 \end{tabular} 
 \end{table}
  \begin{figure}[h]
   \centering
\includegraphics[scale=0.7]{frame.eps}
  \caption{ひずみの計測点(この図は不完全です)}
  \end{figure}

 \section{フレームの変形の様子}
 視覚的に変形の様子が分かりやすいように図を書く.
 変形の様子を分かりやすくするため,変形を何倍か大きくして描く.
 また,その変形の倍率を明記する.
 以下の例は変形を10倍にしたときの図.
  \begin{figure}[h]
\centering
\includegraphics[scale=0.5]{deformedframe.eps}
  \caption{変形の様子}
  \end{figure}
 
\section{感想}
感想を述べる.

\section*{提出場所,締切日}
PandAで確認すること.

\vspace{10\baselineskip}
\section*{\LaTeX のコンパイル方法}
上記ファイルをreport.texとします.以下のように,platexのコマンドを使うとdviファイルができます.
\begin{verbatim}
platex report.tex
\end{verbatim}
dviファイルをpdfに変換します.
\begin{verbatim}
dvipdfmx report.dvi
\end{verbatim}
pdfファイルを見るためには,evinceを使います.
\begin{verbatim}
evince report.pdf
\end{verbatim}

\end{document}